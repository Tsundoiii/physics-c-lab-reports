\documentclass{article}
\usepackage[light, inline-math]{chs-physics-report}
\usepackage{float}
\usepackage{pgfplots}
\usepackage{pgfplotstable}
\usepackage{booktabs}

\pgfplotsset{compat=1.18}

\title{11.3: RC Circuit Lab}
\name{Gavin Chen}
\ww{Daniel Aronov, Cole TerBush, and Tim Marijono}

\begin{document}
\section{Series}
\begin{figure}[H]
    \centering
    \begin{tikzpicture}
        \begin{axis}[enlargelimits=false, xlabel = {Time (s)}, ylabel = {ln(V)}]
            \addplot[domain=0:60, samples=100, blue, thick] {-0.0528*x + 0.494};
            \addlegendentry{\(V = -0.0528t + 0.494\)}

            \addplot+[
                only marks,
            ] table[x=Time (s), y=ln(V), col sep=comma]{Series.csv};
        \end{axis}
    \end{tikzpicture}
    \caption{The total capacitance based on the slope is $86.1\mu F$, which is close to the theoretical value of $76.7\mu F$ for calculated for capacitors in series.}
\end{figure}

\section{Parallel}
\begin{figure}[H]
    \centering
    \begin{tikzpicture}
        \begin{axis}[enlargelimits=false, xlabel = {Time (s)}, ylabel = {ln(V)}]
            \addplot[domain=0:282, samples=100, blue, thick] {-0.00996*x + 0.463};
            \addlegendentry{\(V = -0.00996t + 0.463\)}

            \addplot+[
                only marks,
            ] table[x=Time (s), y=ln(V), col sep=comma]{Parallel.csv};
        \end{axis}
    \end{tikzpicture}
    \caption{The total capacitance based on the slope is $456\mu F$, which is close to the theoretical value of $430\mu F$ for calculated for capacitors in parallel.}
\end{figure}
\end{document}