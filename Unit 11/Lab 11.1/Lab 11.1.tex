\documentclass{article}
\usepackage[light, inline-math]{chs-physics-report}
\usepackage{float}
\usepackage{pgfplots}
\usepackage{pgfplotstable}
\usepackage{booktabs}

\pgfplotsset{compat=1.18}

\title{11.1: Internal Resistance Lab}
\name{Gavin Chen}
\ww{Daniel Aronov}

\begin{document}
\section{Data}
\begin{center}
    \pgfplotstabletypeset[col sep=comma, /pgf/number format/fixed, fixed zerofill, precision=6, every head row/.style={
                before row={
                        \toprule
                    },
                after row={
                        \bottomrule
                    },
            },
        every last row/.style={
                after row=\bottomrule
            }]{Lab 11.1.csv}
\end{center}
\begin{figure}[H]
    \centering
    \begin{tikzpicture}
        \begin{axis}[enlargelimits=false, xlabel = Current ($A$), ylabel = Voltage ($V$)]
            \addlegendentry{$V = -12.5I + 1.$5}

            \addplot[domain=0:0.0254, samples=100, blue, thick] {-12.5*x + 1.5};

            \addplot+[
                only marks,
                mark=*,
                mark options={fill=blue, draw=blue}
            ] table[x=Current (A), y=Voltage (V), col sep=comma]{Lab 11.1.csv};
        \end{axis}
    \end{tikzpicture}
\end{figure}
\section{emf}
The emf we measured was 1.487 $V$.
\section{Internal Resistance}
The internal resistance of the battery is the negative slope of the graph, which is $12.5 \Omega$.
\section{emf Comparison}
The measured emf of 1.487 $V$ was very close to the emf calculated from the $y$-intercept of the graph, which was $1.5 V$.This makes sense, since the $y$-intercept should be exactly the emf.
\end{document}