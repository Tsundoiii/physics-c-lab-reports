\documentclass{article}
\usepackage[light, inline-math]{chs-physics-report}
\usepackage{float}
\usepackage{pgfplots}
\usepackage{pgfplotstable}
\usepackage{booktabs}

\pgfplotsset{compat=1.18}

\title{1.3: Range versus Angle Lab}
\name{Gavin Chen}
\ww{Cole TerBush and Daniel Aronov}

\begin{document}
\section{Graphs}
\begin{figure}[H]
    \centering
    \begin{tikzpicture}
        \begin{axis}[enlarge x limits=false, width=0.6\textwidth, xlabel = Angle (deg), ylabel = Average On-table Range ($m$)]
            \addplot[domain=20:70, samples=100, blue, thick] {-0.166 + 0.0859 * x - 0.000962 * x^2};
            \addlegendentry{$x(\theta) = -0.166 + 0.0859\theta - 0.000962\theta^2$}

            \addplot+[
                only marks,
                mark=*,
                mark options={fill=blue, draw=blue}
            ] table[x=Angle (deg), y=Average On-table Range (m), col sep=comma]{Lab 1.3.csv};
        \end{axis}
    \end{tikzpicture}
    \caption{The on-table ranges for each launch degree, along with an approximating polynomial function.}
    \label{fig:on-table}
\end{figure}

\begin{figure}[H]
    \centering
    \begin{tikzpicture}
        \begin{axis}[enlarge x limits=false, width=0.6\textwidth, xlabel = Angle (deg), ylabel = Average Off-table Range ($m$)]
            \addplot[domain=20:70, samples=100, red, thick] {-0.466 + 0.109 * x - 0.00101 * x^2};
            \addlegendentry{$x(\theta) = -0.466 + 0.109\theta - 0.00101\theta^2$}

            \addplot+[
                only marks,
                mark=*,
                mark options={fill=red, draw=red}
            ] table[x=Angle (deg), y=Average Off-table Range (m), col sep=comma]{Lab 1.3.csv};
        \end{axis}
    \end{tikzpicture}
    \caption{The off-table ranges for each launch degree, along with an approximating polynomial function.}
    \label{fig:off-table}
\end{figure}

\section{First Derivative Test}
\subsubsection*{Style II Launch}
The derivative of the on-table range equation shown in Figure \ref{fig:on-table} is as follows:
\[\frac{dx}{d\theta} = 0.0859 - 0.001924\theta\]
To find the maximum of the function, and thus the greatest range, we set the derivative equal to zero and solve for the angle.
\[0 = 0.0859 - 0.001924\theta\]
\[0.001924\theta = 0.0859\]
\[\theta = \frac{0.0859}{0.001924}\]
\[\theta = 44.65\degree\]
This angle agrees with the angle found visually by looking at Figure \ref{fig:on-table}.
\subsubsection*{Style III Launch}
The derivative of the off-table range equation shown in Figure \ref{fig:off-table} is as follows:
\[\frac{dx}{d\theta} = 0.109 - 0.00202\theta\]
To find the maximum of the function, and thus the greatest range, we set the derivative equal to zero and solve for the angle.
\[0 = 0.109 - 0.00202\theta\]
\[0.00202\theta = 0.109\]
\[\theta = \frac{0.109}{0.00202}\]
\[\theta = 53.96\degree\]
This angle agrees with the angle found visually by looking at Figure \ref{fig:off-table}.

\section{Style II Range Complements}
In any kinematic launch, the distance traveled by an object in the $y$-axis direction is as follows:
\[\Delta y = v_{0_y}t + \frac{1}{2}a_yt^2\]
In a style II launch, $\Delta y$ is 0 and $a_y$ is $-g$. $v_{0_y}$ can also be rewritten as $v_0\sin(\theta)$, where $\theta$ is the angle of launch, giving:
\[0 = v_0\sin(\theta)t - \frac{1}{2}gt^2\]
Factoring out $t$, we obtain:
\[0 = t(v_0\sin(\theta) - \frac{1}{2}gt)\]
There are two possible time solutions for this. One is the trivial solution of 0, but the other is the solution of interest. Solving for that solution, we obtain:
\[t = \frac{2v_0\sin(\theta)}{g}\]
We may now use this time value in the distance equation for the $x$-axis direction:
\[\Delta x = v_{0_x}t + \frac{1}{2}a_xt^2\]
$a_x$ is 0, and $v_{0_x}$ can be rewritten as $v_0\cos(\theta)$, so the equation can be rewritten as:
\[\Delta x = v_0\cos(\theta)t\]
Plugging in the $t$ we obtained earlier, we now obtain:
\[\Delta x = \frac{2v_0^2\sin(\theta)\cos(\theta)}{g}\]
Applying the trigonometric identity $\sin(2\theta) = 2\sin(\theta)\cos(\theta)$, we obtain the range equation:
\[\Delta x = \frac{v_0^2\sin(2\theta)}{g}\]
If an object in a style II launch is launched at a complementary angle, then the angle will be $90\degree - \theta$, which results in:
\[\Delta x = \frac{v_0^2\sin(2(90\degree - \theta))}{g}\]
Which simplifies to:
\[\Delta x = \frac{v_0^2\sin(180\degree - 2\theta)}{g}\]
Since the values of $\sin$ are equivalent for all supplementary angles, $\sin(180\degree - 2\theta)$ is equivalent to $\sin(2\theta)$. Plugging that in, we obtain:
\[\Delta x = \frac{v_0^2\sin(2\theta)}{g}\]
Which is the original range equation. Therefore, any object launched at an angle's complement will have the same range as the same object launched from the original angle.
\end{document}