\documentclass{article}
\usepackage[light, inline-math]{chs-physics-report}
\usepackage{float}
\usepackage{pgfplots}
\usepackage{pgfplotstable}
\usepackage{booktabs}

\pgfplotsset{compat=1.18}

\title{1.2: Acceleration of a Toy Car Lab}
\name{Gavin Chen}
\ww{Cole TerBush and Daniel Aronov}

\begin{document}
\section{Results}
\begin{figure}[H]
    \centering
    \begin{tikzpicture}
        \begin{axis}[enlargelimits=false, xlabel = {Average Time ($s$)}, ylabel = {Distance($m$)}]
            \addplot[domain=0:1.2, samples=100, blue, thick] {-0.00621 + 0.12*x + 0.606*x^2};
            \addlegendentry{\(x(t) = -0.00621 + 0.12t + 0.606t^2\)}

            \addplot+[
                only marks,
            ] table[x=Average Time (s), y=Distance (m), col sep=comma]{Lab 1.2.csv};
        \end{axis}
    \end{tikzpicture}
    \caption{The average time of three trials it took the cart to travel each distance, along with a polynomial approximation $x(t)$ of the position function of the cart.}
    \label{fig:polynomial}
\end{figure}

\section{Plan Statement}
\section{Procedure}
\section{Data}
\pgfplotstabletypeset[col sep=comma, /pgf/number format/fixed, fixed zerofill, precision=6, every head row/.style={
            before row={
                    \toprule
                },
            after row={
                    \bottomrule
                },
        },
    every last row/.style={
            after row=\bottomrule
        }]{Lab 1.2.csv}
\section{Analysis}
\end{document}