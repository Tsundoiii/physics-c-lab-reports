\documentclass{article}
\usepackage[light, inline-math]{chs-physics-report}
\usepackage{float}
\usepackage{pgfplots}
\usepackage{pgfplotstable}
\usepackage{booktabs}

\pgfplotsset{compat=1.18}

\title{1.2: Acceleration of a Toy Car Lab}
\name{Gavin Chen}
\ww{Cole TerBush and Daniel Aronov}

\begin{document}
\section{Plan Statement}
We will record the car traveling straight along a meterstick and record how long it takes to travel certain distances, then approximate a position function using that data, then differentiate that function to determine whether the car acceleration is constant or not.
\section{Procedure}
\begin{enumerate}
    \item Obtain car, meterstick, and slow-motion camera.
    \item Pull car back a certain distance.
    \item Start recording car with camera.
    \item Release car, and record car with camera until car stops.
    \item Examine recording to determine how long it took car to reach distances of 0.05 $m$ increments.
    \item Repeat with different pullback distances.
\end{enumerate}
\section{Data}
\begin{figure}[H]
    \centering
    \begin{tikzpicture}
        \begin{axis}[enlargelimits=false, xlabel = Trial 1 Time ($s$), ylabel = Distance($m$)]
            \addplot[domain=0:0.35, samples=100, blue, thick] {0.0387 + 2.15 * x - 1.15 * x^2};
            \addlegendentry{Trial 1 ($R^2 = 0.996$)}
            \addplot[domain=0:0.35, samples=100, red, thick] {0.0383 + 0.932 * x + 5.32 * x^2};
            \addlegendentry{Trial 2 ($R^2 = 0.998$)}
            \addplot[domain=0:0.35, samples=100, green, thick] {0.035 + 0.799 * x + 7.07 * x^2};
            \addlegendentry{Trial 3 ($R^2 = 0.998$)}

            \addplot+[
                only marks,
                mark=*,
                mark options={fill=blue, draw=blue}
            ] table[x=Trial 1 Time (s), y=Distance (m), col sep=comma]{Lab 1.2.csv};
            \addplot+[
                only marks,
                mark=*,
                mark options={fill=red, draw=red}
            ] table[x=Trial 2 Time (s), y=Distance (m), col sep=comma]{Lab 1.2.csv};
            \addplot+[
                only marks,
                mark=*,
                mark options={fill=green, draw=green}
            ] table[x=Trial 3 Time (s), y=Distance (m), col sep=comma]{Lab 1.2.csv};
        \end{axis}
    \end{tikzpicture}
    \caption{The position functions of each trial, along with the associated $R^2$ values.}
    \label{fig:times}
\end{figure}

\pgfplotstabletypeset[col sep=comma, /pgf/number format/fixed, fixed zerofill, precision=2, every head row/.style={
            before row={
                    \toprule
                },
            after row={
                    \bottomrule
                },
        },
    every last row/.style={
            after row=\bottomrule
        }]{Lab 1.2.csv}
\section{Analysis}
All three trials were modeled with second-degree polynomial position functions as follows:
\[x_1(t) = 0.0387 + 2.15x - 1.15x^2\]
\[x_2(t) = 0.0383 + 0.932x + 5.32x^2\]
\[x_3(t) = 0.035 + 0.799x + 7.07x^2\]
These functions are very accurate models of the data, as indicated by the very high $R^2$ values of each of these functions when compared with the discreet data they respectively model.

The acceleration functions derived from these functions are as follows:
\[a_1(t) = -2.30\]
\[a_2(t) = 10.64\]
\[a_3(t) = 14.14\]
Since these are all constant functions, we have demonstrated that the manufacturer's claim of the car experiencing near constant acceleration is supported.
\end{document}