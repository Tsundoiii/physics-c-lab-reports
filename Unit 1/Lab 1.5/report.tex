\documentclass{article}
\usepackage[light, inline-math]{chs-physics-report}
\usepackage{float}
\usepackage{pgfplots}
\usepackage{pgfplotstable}
\usepackage{booktabs}

\pgfplotsset{compat=1.18}

\title{1.4: Range versus Angle Lab}
\name{Gavin Chen}
\ww{Cole TerBush and Daniel Aronov}

\begin{document}
\section{Equation}
Beginning with the equation:
\[\Delta x = v_0\cos(\theta)t + \frac{1}{2}a_xt^2\]
$a_x$ is 0, so that term disappears. Solving for $t$, we obtain:
\[t = \frac{\Delta x}{v_0\cos(\theta)}\]
We may plug that into the height equation:
\[\Delta y = v_0\sin(\theta)t + \frac{1}{2}a_yt^2\]
To obtain:
\[\Delta y = v_0\sin(\theta)(\frac{\Delta x}{v_0\cos(\theta)}) + \frac{1}{2}a_y(\frac{\Delta x}{v_0\cos(\theta)})^2\]
\[\Delta y = \Delta x\tan(\theta) + \frac{a_y(\Delta x^2)\sec^2(\theta)}{2v_0^2}\]
\section{Conclusion and Error Analysis}
\end{document}