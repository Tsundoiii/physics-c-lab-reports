\documentclass{article}
\usepackage[light, inline-math]{chs-physics-report}
\usepackage{float}
\usepackage{pgfplots}
\usepackage{pgfplotstable}
\usepackage{booktabs}

%\usepgfplotslibrary{external}
%\tikzexternalize
\pgfplotsset{compat=1.18}

\title{1.1: Cart on Ramp Lab}
\name{Gavin Chen}
\ww{Cole TerBush and Daniel Aronov}

\begin{document}
\section{Results}
\begin{figure}[H]
    \centering
    \begin{tikzpicture}
        \begin{axis}[enlargelimits=false, xlabel = {Average Time ($s$)}, ylabel = {Distance($m$)}]
            \addplot[domain=0:1.2, samples=100, blue, thick] {-0.00621 + 0.12*x + 0.606*x^2};
            \addlegendentry{\(x(t) = -0.00621 + 0.12t + 0.606t^2\)}

            \addplot+[
                only marks,
            ] table[x=Average Time (s), y=Distance (m), col sep=comma]{Lab 1.1.csv};
        \end{axis}
    \end{tikzpicture}
    \caption{The average time of three trials it took the cart to travel each distance, along with a polynomial approximation $x(t)$ of the position function of the cart.}
    \label{fig:polynomial}
\end{figure}

\begin{figure}[H]
    \centering
    \begin{tikzpicture}
        \begin{axis}[enlargelimits=false, xlabel = {Average Time Squared ($s^2$)}, ylabel = {Distance($m$)}]
            \addplot[domain=0:1.415136, samples=100, blue, thick] {0.682*x + 0.0362};
            \addlegendentry{\(\Delta x = 0.682x + 0.0362\)}

            \addplot+[
                only marks,
            ] table[x=Average Time Squared ($s ^ 2$), y=Distance (m), col sep=comma]{Lab 1.1.csv};
        \end{axis}
    \end{tikzpicture}
    \caption{The average time squared of three trials it took the cart to travel each distance, along with a linear approximation of the graph.}
    \label{fig:linear}
\end{figure}

\section{Analysis}
\subsection{Slope of Linearized Position Graph}
The kinematics equation most suited to being linearized to use to find accerelation is as follows:
\[\Delta x = v_0t + \frac{1}{2}at^2\]
Since our starting point was defined as our zero point and initial velocity $v_0$ was close to zero, the equation can be rewritten as a position function of time $t$ as follows:
\[\Delta x = \frac{1}{2}at^2\]
By graphing $t^2$ on the $x$-axis and $\Delta x$ on the $y$-axis, as is done in Figure \ref{fig:linear}, the slope of the graph is $\frac{1}{2}a$, so doubling the slope would result in the value of $a$. The slope of the graph in Figure \ref{fig:linear} is $0.682 \frac{m}{s ^ 2}$, thus the accerelation derived from that graph is $1.364 \frac{m}{s ^ 2}$.

\subsection{Derivatives of Polynomial Position Function}
The polynomial approximation of the position function is as follows:
\[x(t) = -0.00621 + 0.12t + 0.606t^2\]
Taking the first derivative, we obtain the velocity function:
\[v(t) = 0.12 + 1.212t\]
Taking the second derivative, we obtain the acceleration function:
\[a(t) = 1.212\]
The acceleration function is constant, meaning the cart experienced a uniform acceleration of $1.212 \frac{m}{s ^ 2}$ as it traveled along the ramp.

\subsection{Derivatives of Linearized vs. Polynomial Position Functions}
The accerelation obtained using the linearized graph was $1.364 \frac{m}{s ^ 2}$, and the acceleration obtained from the polynomial position graph was $1.212 \frac{m}{s ^ 2}$. These numbers are fairly close, indicating the value of the cart's acceleration along the ramp is close to those quantities.

\subsection{Possible Improvements to Lab}
If we were to redo our lab to improve our results, there would be a few improvements we would make to our methodology to achieve that goal. Firstly, we would strive for more consistency in the release of the cart on the ramp. A few times, our cart was a little farther behind the photogate than we would've hoped for, meaning the cart did have slight velocity when passing through it. There were also some instances where a little force was applied to the cart, also slightly increasing its starting velocity, so reducing that would also be a goal if this lab were to be repeated. Collecting more data points per distance would also be helpful in increasing the accuracy of our acceleration values.


\appendix
\section{Data}
\pgfplotstabletypeset[col sep=comma, /pgf/number format/fixed, fixed zerofill, precision=6, every head row/.style={
            before row={
                    \toprule
                },
            after row={
                    \bottomrule
                },
        },
    every last row/.style={
            after row=\bottomrule
        }]{Lab 1.1.csv}
\end{document}