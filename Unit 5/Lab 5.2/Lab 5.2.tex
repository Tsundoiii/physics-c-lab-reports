\documentclass{article}
\usepackage[light, inline-math]{chs-physics-report}
\usepackage{float}
\usepackage{pgfplots}
\usepackage{pgfplotstable}
\usepackage{booktabs}

\pgfplotsset{compat=1.18}

\title{5.2: Moment of Inertia Lab}
\name{Gavin Chen}
\ww{Cole TerBush and Tim Marijono}

\begin{document}
\section{Derivation}
\[U = K_{mech} + K_{rot}\]
\[mgh = \frac{1}{2}mv^2 + \frac{1}{2}I\omega^2\]
\[mgh = \frac{1}{2}mv^2 + \frac{1}{2}I\frac{v^2}{r^2}\]
\[mgh - \frac{1}{2}mv^2 = \frac{1}{2}I\frac{v^2}{r^2}\]
By graphing $mgh - \frac{1}{2}mv^2$ on the $y$-axis and $\frac{v^2}{r^2}$ on the $x$-axis, the constant slope of $\frac{1}{2}I$ can be obtained, which can be doubled to obtain $I$.

Having measured the radius and the mass, the theoretical moment of inertia was found using $I = \frac{2}{5}mr^2$:
\begin{center}
    \pgfplotstabletypeset[col sep=comma, precision=2, every head row/.style={
                before row={
                        \toprule
                    },
                after row={
                        \bottomrule
                    },
            },
        every last row/.style={
                after row=\bottomrule
            }]{Lab 5.2 Constants.csv}
\end{center}
\section{Procedure}
\begin{enumerate}
    \item Mass ball and measure ball radius.
    \item Attach one end of ramp to pole.
    \item Release ball and record time for ball to reach ground.
    \item Repeat for three trials at same pole attachment height to obtain average time.
    \item Repeat for four different pole attachment heights.
\end{enumerate}
\section{Results}
\begin{figure}[H]
    \centering
    \begin{tikzpicture}
        \begin{axis}[enlargelimits=false, xlabel = {$\omega^2$ $(\frac{1}{s^2})$}, ylabel = {$mgh - \frac{1}{2}mv^2$ $(\frac{kg \cdot m^2}{s^2})$}]
            \addplot[domain=0:60375.35338, samples=100, blue, thick] {0.00000135*x - 0.0294};
            \addplot+[
                only marks,
            ] table[x=Angular Velocity Squared, y=y, col sep=comma]{Lab 5.2.csv};
        \end{axis}
    \end{tikzpicture}
\end{figure}
The slope of the graph is $1.35 \cdot 10^{-6} \frac{kg \cdot m^2}{s^2}$, which can be doubled to obtain an $I$ value of $2.70 \cdot 10^{-6} \frac{kg \cdot m^2}{s^2}$, which is within an order of magnitude of the theoretical moment of inertia.
\end{document}