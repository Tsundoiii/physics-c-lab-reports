\documentclass{article}
\usepackage[light, inline-math]{chs-physics-report}
\usepackage{pgfplots}

\usepgfplotslibrary{external}
\tikzexternalize
\pgfplotsset{compat=1.18}

\title{Cart on Ramp Lab}
\name{Gavin Chen}
\ww{Cole TerBush and Daniel Aronov}

\begin{document}
\section{Results}
\begin{figure}
    % \begin{tikzpicture}
    %     \begin{axis}[enlargelimits=false]
    %         \addplot+[
    %             only marks,
    %             mark=o
    %         ]
    %         table[x=Average Time (s), y=Distance (m), col sep=comma]{Lab 1.1.csv}
    %     \end{axis}
    % \end{tikzpicture}
    \label{fig:linear}
\end{figure}
\section{Analysis}
\subsection{Slope of Linearized Position Graph}
The kinematics equation most suited to being linearized to use to find accerelation is as follows:
\[\Delta x = v_0t + \frac{1}{2}at^2\]
Since our starting point was defined as our zero point and initial velocity $v_0$ was close to zero, the equation can be rewritten as a position function of time $t$ as follows:
\[x(t) = \frac{1}{2}at^2\]
By graphing $t^2$ on the $x$-axis and $x(t)$ on the $y$-axis, as is done in Figure \ref{fig:linear}, the slope of the graph is $\frac{1}{2}a$, so doubling the slope would result in the value of $a$. The slope of the graph in Figure \ref{fig:linear} is $0.682$, thus the accerelation derived from that graph is $1.364$.

\subsection{Derivatives of Polynomial Position Function}
\subsection{Derivatives of Linearized vs. Polynomial Position Functions}

\subsection{Possible Improvements to Lab}
If we were to redo our lab to improve our results, there would be a few improvements we would make to our methodology to achieve that goal. Firstly, we would strive for more consistency in the release of the cart on the ramp. A few times, our cart was a little farther behind the photogate than we would've hoped for, meaning the cart did have slight velocity when passing through it. There were also some instances where a little force was applied to the cart, also slightly increasing its starting velocity, so reducing that would also be a goal if this lab were to be repeated. Collecting more data points per distance would also be helpful in increasing the accuracy of our acceleration values.


\appendix
\section{Data}
\end{document}