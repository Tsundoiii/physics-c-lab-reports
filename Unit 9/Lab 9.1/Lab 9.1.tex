\documentclass{article}
\usepackage[light, inline-math]{chs-physics-report}
\usepackage{float}
\usepackage{pgfplots}
\usepackage{pgfplotstable}
\usepackage{booktabs}

\pgfplotsset{compat=1.18}

\title{6.2: Springs in Series and Parallel Lab}
\name{Gavin Chen}
\ww{Cole TerBush, Daniel Aronov, and Tim Marijono}

\begin{document}
\section{Data}
\subsection{Experimental Values}
\begin{center}
    \pgfplotstabletypeset[col sep=comma, /pgf/number format/fixed, fixed zerofill, precision=3, every head row/.style={
                before row={
                        \toprule
                    },
                after row={
                        \bottomrule
                    },
            },
        every last row/.style={
                after row=\bottomrule
            }]{Voltages.csv}
\end{center}
\subsection{Theoretical Values}
\begin{center}
    \pgfplotstabletypeset[col sep=comma, /pgf/number format/fixed, fixed zerofill, precision=3, every head row/.style={
                before row={
                        \toprule
                    },
                after row={
                        \bottomrule
                    },
            },
        every last row/.style={
                after row=\bottomrule
            }]{Theoretical Voltages.csv}
\end{center}
\section{Analysis}
\subsection{Plot}
\begin{figure}[H]
    \centering
    \begin{tikzpicture}
        \begin{axis}[enlargelimits=false, xlabel = r ($m$), ylabel = Voltage ($V$)]
            \addlegendentry{Experimental}
            \addlegendentry{Theoretical Sphere}
            \addlegendentry{Theoretical Cylinder}

            \addplot+[
                only marks,
                mark=*,
                mark options={fill=blue, draw=blue}
            ] table[x=r (m), y=Average (V), col sep=comma]{Voltages.csv};
            \addplot+[
                only marks,
                mark=*,
                mark options={fill=red, draw=red}
            ] table[x=r (m), y=Sphere (V), col sep=comma]{Theoretical Voltages.csv};
            \addplot+[
                only marks,
                mark=*,
                mark options={fill=green, draw=green}
            ] table[x=r (m), y=Cylinder (V), col sep=comma]{Theoretical Voltages.csv};
        \end{axis}
    \end{tikzpicture}
\end{figure}
\subsection{Geometry}
The geometry of the dot-ring configuration is closest to a cylinder, because the experimental voltages measured on the dot-ring configuration more closely match the theoretical voltages of a cylinderical geometry rather than a spherical geometry.
\subsection{Improvements}
If we were to repeat the lab, we would be more precise in where we measured our voltages, since we were not entirely consistent with that when we did our lab.
\subsection{$\epsilon_0$ Value}
The average experimental $\epsilon_0$ value obtained was $6.32 * 10^-12 \frac{C^2}{N * m^2}$, which has a percent error of $28.59\%$
\end{document}