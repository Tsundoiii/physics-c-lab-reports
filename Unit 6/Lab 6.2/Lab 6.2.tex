\documentclass{article}
\usepackage[light, inline-math]{chs-physics-report}
\usepackage{float}
\usepackage{pgfplots}
\usepackage{pgfplotstable}
\usepackage{booktabs}

\pgfplotsset{compat=1.18}

\title{6.2: Springs in Series and Parallel Lab}
\name{Gavin Chen}
\ww{Cole TerBush and Daniel Aronov}

\begin{document}
\section{Hypotheses}
The spring constant will remain the same when the springs are in parallel.

\noindent
The spring constant will be halved when the springs are in series.
\section{Procedure}
\subsection{Parallel}
\begin{enumerate}
    \item Hang two identical springs from the spring brakcet.
    \item Hang a mass from the two springs.
    \item Stretch mass on springs.
    \item Record time for 10 oscillations of mass.
    \item Divide by 10 to find period.
    \item Repeat with nonidentical springs.
\end{enumerate}
\subsection{Series}
\begin{enumerate}
    \item Hang one spring on spring bracket, then hang another identical spring on the spring.
    \item Hang a mass from the bottom of second spring.
    \item Stretch mass on springs.
    \item Record time for 10 oscillations of mass.
    \item Divide by 10 to find period.
    \item Repeat with nonidentical springs.
\end{enumerate}
\section{Data}
\section{Derivations}
\subsection{Experimental $k$}
\[T = \tau\sqrt{\frac{m}{k}}\]
\[T^2 = \frac{m}{k}\]
\[k = \frac{m}{T^2}\]
\subsection{Parallel $k$}
\[k_1x + k_2x = F_g\]
\[k_1 + k_2 = \frac{F_g}{x}\]
\[k_{eff}x = F_g\]
\[k_{eff} = \frac{F_g}{x}\]
\[k_{eff} = k_1 + k_2\]
\subsection{Series $k$}
\[k_1x_1 = k_2x_2 = F_g\]
\[x_1 = \frac{F_g}{k_1}\]
\[x_2 = \frac{F_g}{k_2}\]
\[k_{eff}(x_1 + x_2) = F_g\]
\[k_{eff}(\frac{F_g}{k_1} + \frac{F_g}{k_2}) = F_g\]
\[k_{eff}(\frac{1}{k_1} + \frac{1}{k_2}) = 1\]
\[k_{eff} = \frac{1}{\frac{1}{k_1} + \frac{1}{k_2}}\]
\section{Conclusion}
\end{document}