\documentclass{article}
\usepackage[light, inline-math]{chs-physics-report}
\usepackage{float}
\usepackage{pgfplots}
\usepackage{pgfplotstable}
\usepackage{booktabs}

\pgfplotsset{compat=1.18}

\title{12.2: Slinky Solenoid Lab}
\name{Gavin Chen}

\begin{document}
\section{Part 1}
\subsection{Data}
\begin{figure}[H]
    \centering
    \begin{tikzpicture}
        \begin{axis}[enlargelimits=false, xlabel = {Current in Solenoid (A)}, ylabel = {Magnetic Field (T)}]
            \addplot[domain=0:2, samples=100, blue, thick] {9.67e-5*x + 1.11e-6};

            \addplot+[
                only marks,
            ] table[x=Current in Solenoid (A), y=Magnetic Field (T), col sep=comma]{Part 1.csv};
        \end{axis}
    \end{tikzpicture}
    \caption{Magnetic field vs. current in solenoid, with a solenoid length of 1 $m$ with 82 turns.}
\end{figure}
\subsection{Analysis}
\begin{figure}[H]
    \centering
    \begin{tikzpicture}
        \begin{axis}[enlargelimits=false, xlabel = {Turns Per Meter}, ylabel = {Magnetic Field (T)}]
            \addplot[domain=0:328, samples=100, blue, thick] {2.49e-6*x - 2.67e-5};

            \addplot+[
                only marks,
            ] table[x=Turns Per Meter, y=Magnetic Field (T), col sep=comma]{Part 2.csv};
        \end{axis}
    \end{tikzpicture}
    \caption{Magnetic field vs. current in solenoid, with a solenoid length of 1 $m$ with 82 turns.}\
\end{figure}
\end{document}