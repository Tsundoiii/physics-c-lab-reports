\documentclass{article}
\usepackage[light, inline-math]{chs-physics-report}
\usepackage{float}
\usepackage{pgfplots}
\usepackage{pgfplotstable}
\usepackage{booktabs}

\pgfplotsset{compat=1.18}

\title{12.2: Slinky Solenoid Lab}
\name{Gavin Chen}
\ww{Cole TerBush and Daniel Aronov}

\begin{document}
\section{Part 1}
\subsection{Data}
\begin{center}
    \pgfplotstabletypeset[col sep=comma, fixed zerofill, precision=2, every head row/.style={
                before row={
                        \toprule
                    },
                after row={
                        \bottomrule
                    },
            },
        every last row/.style={
                after row=\bottomrule
            }]{Part 1.csv}
\end{center}
\subsection{Analysis}
\begin{figure}[H]
    \centering
    \begin{tikzpicture}
        \begin{axis}[enlargelimits=false, xlabel = {Current in Solenoid (A)}, ylabel = {Magnetic Field (T)}]
            \addplot[domain=0:2, samples=100, blue, thick] {9.67e-5*x + 1.11e-6};

            \addplot+[
                only marks,
            ] table[x=Current in Solenoid (A), y=Magnetic Field (T), col sep=comma]{Part 1.csv};
        \end{axis}
    \end{tikzpicture}
    \caption{Magnetic field vs. current in solenoid, with a solenoid length of 1 $m$ with 82 turns. The shape of the graph and the equation for magnetic field in a solenoid roughly agree with each other. The slope of the graph is the permeability constant $\mu_0$ times the number of turns divided by the length of the solenoid, which here is $9.67 * 10^{-5} \frac{T}{A}$. Using the slope to calculate $\mu_0$, we get $1.18 \cdot 10^{-6} \frac{kg \cdot m}{s^2 \cdot A^2}$, with a $6.16\%$ error from the true value.}
\end{figure}
\section{Part 2}
\subsection{Data}
\begin{center}
    \pgfplotstabletypeset[col sep=comma, fixed zerofill, precision=2, every head row/.style={
                before row={
                        \toprule
                    },
                after row={
                        \bottomrule
                    },
            },
        every last row/.style={
                after row=\bottomrule
            }]{Part 2.csv}
\end{center}
\subsection{Analysis}
\begin{figure}[H]
    \centering
    \begin{tikzpicture}
        \begin{axis}[enlargelimits=false, xlabel = {Turns Per Meter}, ylabel = {Magnetic Field (T)}]
            \addplot[domain=0:328, samples=100, blue, thick] {2.49e-6*x - 2.67e-5};

            \addplot+[
                only marks,
            ] table[x=Turns Per Meter, y=Magnetic Field (T), col sep=comma]{Part 2.csv};
        \end{axis}
    \end{tikzpicture}
    \caption{Magnetic field vs. turns per meter of a solenoid, with a solenoid with 82 turns with a current of $2 A$ running through it. The shape of the graph and the equation for magnetic field in a solenoid roughly agree with each other. The slope of the graph is the permeability constant $\mu_0$ times the current through the solenoid, which here is $2.49 * 10^{-6} \frac{T \cdot m}{turns}$. Using the slope to calculate $\mu_0$, we get $1.25 \cdot 10^{-6} \frac{kg \cdot m}{s^2 \cdot A^2}$, with a $0.53\%$ error from the true value.}
\end{figure}
\section{Analysis}
The average value of $\mu_0$ is $1.21 \cdot 10^{-6} \frac{kg \cdot m}{s^2 \cdot A^2}$, with a $3.54\%$ error.
\end{document}